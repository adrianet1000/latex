\section{CARACTERES PROHIBIDOS}

Hay unos cuantos caracteres que tienen un
significado particular en \LaTeX~y, por lo tanto,
no los podemos usar directamente en nuestros documentos.

\begin{verbatim}
\
%
{
}
$
#
&
^
~
_
\end{verbatim}

Si queremos escribir un backslash podemos escribirlo invocando su codigo

\begin{verbatim}
\textbackslash
\end{verbatim}

Luego los demas simbolos los escribimos anteponiendo el simbolo de backslash.

Esta información fue extraida de la pagina web.

\url{http://elclubdelautodidacta.es/wp/2011/08/latex-capitulo-10-caracteres-prohibidos/}

El simbolo de la tilde \~{} genera un espacio en blanco. Para generar ese simbolo
debe agregarse el simbolo backslash antes de escribir la tilde.

\section{error de coloreado de syntaxis en el entorno lstlisting}

cuando estamos en el entorno lstlisting por ejemplo haciendo un ejemplo de
codigo de csharp luego de hacer un comenario seguido del simbolo de dolar o lo que
se conoce como plantilla literales, luego el texto en latex dentro del
editor vim dentro de windows terminal previou nos colorea todo como texto
a partir del simbolo de dolar. una solución es luego finalizado el entorno
lstlisting poner inmediatamente como comentario de latex lo siguiente

\begin{verbatim}
%stopzone
\end{verbatim}

\section{syntax en latex}

Una forma de saber la sintaxis que se se esta utilizando en vim para colorear
es invocando la configuración syntax de la siguiente manera.

:set syntax?

esto nos dira con que sintaxis se esta pintando el codigo que estamos
escribiendo

:set syntax=tex

Esta intrucción pintara el codigo en codigo latex
